\documentclass[12pt]{article}

\usepackage{amssymb,amsmath, amsfonts,eurosym,geometry,ulem,graphicx,caption,subcaption,color,setspace,sectsty,comment,footmisc,natbib,pdflscape,array,hyperref, mathtools,bbm, longtable}


% \usepackage[nomarkers,notablist,nofiglist]{endfloat}
\usepackage{threeparttable}
\usepackage{adjustbox}
\usepackage{booktabs}
\usepackage{tikz}
% \renewcommand{\efloatseparator}{\medskip}
\usepackage{fancyhdr}
\usepackage[ruled, vlined]{algorithm2e}
\usepackage{float}
\usepackage{appendix}
\linespread{1.5}
\normalem

% \onehalfspacing
% \newtheorem{theorem}{Theorem}
% \newtheorem{corollary}[theorem]{Corollary}
% \newtheorem{prop}{Proposition}
% \newenvironment{proof}[1][Proof]{\noindent\textbf{#1.}}{\ \rule{0.5em}{0.5em}}

% \newtheorem{hyp}{Hypothesis}
% \newtheorem{subhyp}{Hypothesis}[hyp]
% \renewcommand{\thesubhyp}{\thehyp\alph{subhyp}}
\usepackage{amsthm}
\onehalfspacing
\newtheorem{theorem}{Theorem}
\newtheorem{corollary}[theorem]{Corollary}
\newtheorem{prop}{Proposition}
\newtheorem{hyp}{Hypothesis}
\newtheorem{subhyp}{Hypothesis}[hyp]
\renewcommand{\thesubhyp}{\thehyp\alph{subhyp}}

\newcommand{\red}[1]{{\color{red} #1}}
\newcommand{\blue}[1]{{\color{blue} #1}}

\newcolumntype{L}[1]{>{\raggedright\let\newline\\arraybackslash\hspace{0pt}}m{#1}}
\newcolumntype{C}[1]{>{\centering\let\newline\\arraybackslash\hspace{0pt}}m{#1}}
\newcolumntype{R}[1]{>{\raggedleft\let\newline\\arraybackslash\hspace{0pt}}m{#1}}

\geometry{left=1.0in,right=1.0in,top=1.0in,bottom=1.0in}

\begin{document}
\begin{titlepage}
\title{Cultural Transmission Models}
\author{Harlly Zhou}
\date{\today}
\maketitle
\begin{abstract}
\noindent
\cite{bisinEconomicModelsCultural2025} summarizes recent advances in modelling cutural transmission in the economics literature. I would like to depart from here to investigate what are possible path for future research topic and question.
\bigskip
\end{abstract}
\setcounter{page}{0}
\thispagestyle{empty}
\end{titlepage}
\pagebreak \newpage

\section{Canonical Model of Cultural Transmission}
\subsection{Transmission Technology}
Assume that cultural transmission is intergenerational, and there are only two cultural traits $i$ and $j$. Assume that population size is normalized to 1. There are two channels for intergenerational cultural transmission: \textit{direct vertical} or \textit{parental} transmission, and \textit{oblique} or \textit{social} transmission. For the first channel, there is a probability of $d^i$ for a child to be socialized to cultural trait $i$ by a trait $i$ parent. The residual probability $1-d^i$ is for the second channel. A child is randomly matched to a member of the parental generation and is socialized to his/her trait. Conditioanl on being socialized via the social transmission channel (with probability $1-d^i$), the probability of a child to be socialized to trait $i$ is $q^i$. \textbf{Note that if a child's parent is of trait $i$, then the child can ONLY be socialized to trait $i$, while she can be socialized to $j\neq i$ only via the social channel.}

Let $P^{ij}$ denote the transition probability that a child is socialized to trait $j$ with trait $i$ parent. Then the cultural transmission technology is represented by
\begin{align}
    P^{ii} = d^i + (1-d^i)q^i,\qquad \qquad \qquad P^{ij} = (1-d^i)(1-q^i).\label{eq:inter_transition_prob}
\end{align}
Then the discrete time dynamics, by Law of Large numbers, will be
\begin{align*}
    \Delta q_{t+1}^{i} = (1-q_t^i)P_t^{ji} - q_t^i P_t^{ij}.
\end{align*}
In continuous time, we get the \textit{logistic equation}:
\begin{align}
    \dot{q}^i = q^i (1-q^i) (d^i-d^j),\label{eq:logistic_eq}
\end{align}
where $d^i-d^j$ reflects the difference in parental rates. Define $f^i\equiv d^i - d^j$ to be the \textit{relative cultural fitness}.

\subsection{Choice}
A parent with trait $i$ get payoff $V^{ij}$ if her child acquires trait $j$. Assume that $V^{ii}>V^{ij}$. A rational parent chooses the transmission rate $d^i$, facing a cost of $c(d^i)$ which increases with $d^i$ and is convex. The expected parental payoff is $P^{ii}V^{ii}+P^{ij}V^{ij}$, where $P^{ii}$ and $P^{ij}$ are defined by \eqref{eq:inter_transition_prob}. Note that this can be reduced to $V^{ij}+P^{ii}\Delta V^i$, where $\Delta V^i = V^{ii} - V^{ij}$. The socialization choice problem is reduced to
\begin{align}
    \max_{d^i} \qquad& P^{ii} \Delta V^i - c(d^i) \label{opt:socialization_choice} \\
    \text{s.t.} \qquad & P^{ii} = d^i + (1-d^i) q^i. \nonumber
\end{align}
The solution is $d^i = (c')^{(-1)}\left((1-q^i)\Delta V^i\right)$. It increases with $\Delta V^i$ and decreases with $q^i$. 

Given this socialization choice setting, we can maintain a steady state / a stable equilibrium where
\begin{align}
    0 < q^i < 1 \text{such that}\ f^i(q^i)=0. \label{eq:eqm_cond}
\end{align}




\setlength\bibsep{0pt}
\bibliographystyle{aer.bst}
\bibliography{library.bib}

\end{document}